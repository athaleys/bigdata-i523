\documentclass[sigconf]{acmart}

\usepackage{hyperref}

\usepackage{endfloat}
\renewcommand{\efloatseparator}{\mbox{}} % no new page between figures

\usepackage{booktabs} % For formal tables

\settopmatter{printacmref=false} % Removes citation information below abstract
\renewcommand\footnotetextcopyrightpermission[1]{} % removes footnote with conference information in first column
\pagestyle{plain} % removes running headers

\begin{document}
\title{Big Data Application in Restaurant Industry}


\author{Sushant Athaley}
\affiliation{%
  \institution{Indiana University}
}
\email{sathaley@iu.edu}



\begin{abstract}
This paper provides insight into how big data can be used in the restaurant industry. It will also explore how big data can be collected and analyzed so that it helps restaurant industry to do better in profit margins and give their customer a great hospitality experience. Paper will try to find out current technologies and solutions available in big data processing for the restaurant industry. It will also focus on various challenges involved in using big data in the restaurant business. This paper is a review/research paper which considers information from various sources like articles, books and web to provide the information. 
\end{abstract}

\keywords{big data, restaurant, application, analytics}


\maketitle

\section{Introduction}
Big data is revolutionizing the way business is getting conducted in various industries. The retailer like Amazon uses it to provide personalized buying suggestions and social networking site like LinkedIn uses it to connect more people. Question is, do we have big data available for the restaurant industry and how big data application is going to be beneficial. The restaurant industry is facing challenges like shrinking labor pool, moderate economic growth, costly labor, challenging profit margin, high competition, moderate sales growth and growing expectation from the customer on the dining experience, can big data application help overcome these challenges.\cite{www-restaurant-challenges}


\section{Big Data for Restaurant}

\section{Collect Big Data}

\section{Big Data Analytics}

\section{Solution and Tools Available}

\section{Challenges of Using Big Data}

\section{Conclusions}

This paragraph will end the body of this sample document.  Remember
that you might still have Acknowledgments or Appendices; brief samples
of these follow.  There is still the Bibliography to deal with; and we
will make a disclaimer about that here: with the exception of the
reference to the \LaTeX\ book, the citations in this paper are to
articles which have nothing to do with the present subject and are
used as examples only.



\appendix

%Appendix A
\section{Headings in Appendices}

The rules about hierarchical headings discussed above for the body of
the article are different in the appendices.  In the \textbf{appendix}
environment, the command \textbf{section} is used to indicate the
start of each Appendix, with alphabetic order designation (i.e., the
first is A, the second B, etc.) and a title (if you include one).  So,
if you need hierarchical structure \textit{within} an Appendix, start
with \textbf{subsection} as the highest level. Here is an outline of
the body of this document in Appendix-appropriate form:

\subsection{Introduction}
\subsection{The Body of the Paper}
\subsubsection{Type Changes and  Special Characters}
\subsubsection{Math Equations}
\paragraph{Inline (In-text) Equations}
\paragraph{Display Equations}
\subsubsection{Citations}
\subsubsection{Tables}
\subsubsection{Figures}
\subsubsection{Theorem-like Constructs}
\subsubsection*{A Caveat for the \TeX\ Expert}
\subsection{Conclusions}
\subsection{References}

Generated by bibtex from your \texttt{.bib} file.  Run latex, then
bibtex, then latex twice (to resolve references) to create the
\texttt{.bbl} file.  Insert that \texttt{.bbl} file into the
\texttt{.tex} source file and comment out the command
\texttt{{\char'134}thebibliography}.

% This next section command marks the start of
% Appendix B, and does not continue the present hierarchy

\section{More Help for the Hardy}

Of course, reading the source code is always useful.  The file
\path{acmart.pdf} contains both the user guide and the commented code.

\begin{acks}

  The authors would like to thank Dr. Yuhua Li for providing the
  matlab code of the \textit{BEPS} method.

  The authors would also like to thank the anonymous referees for
  their valuable comments and helpful suggestions. The work is
  supported by the \grantsponsor{GS501100001809}{National Natural
    Science Foundation of
    China}{http://dx.doi.org/10.13039/501100001809} under Grant
  No.:~\grantnum{GS501100001809}{61273304}
  and~\grantnum[http://www.nnsf.cn/youngscientsts]{GS501100001809}{Young
    Scientsts' Support Program}.

\end{acks}

\bibliographystyle{ACM-Reference-Format}
\bibliography{report} 

\end{document}
